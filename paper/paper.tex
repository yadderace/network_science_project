\documentclass[10pt,twocolumn]{article}

% ------------------------
% PACKAGES
% ------------------------
\usepackage[utf8]{inputenc}
\usepackage{graphicx}
\usepackage{amsmath}
\usepackage{amsfonts}
\usepackage{amssymb}
\usepackage{cite}
\usepackage{hyperref}
\usepackage{geometry}
\usepackage{titlesec}
\usepackage{lipsum}

% Page margins
\geometry{margin=0.75in}

% Hyperlink setup
\hypersetup{
    colorlinks=true,
    linkcolor=blue,
    citecolor=blue,
    urlcolor=blue
}

% Title formatting
\titleformat{\section}{\large\bfseries}{\thesection}{1em}{}
\titleformat{\subsection}{\normalsize\bfseries}{\thesubsection}{1em}{}

% ------------------------
% TITLE
% ------------------------
\title{ \bfseries Title of Your Paper Goes Here \\[0.2cm]
\large Identifying Key Species in a Food Web using Network Analysis}

\author{
    Yadder Joshua Aceituno González \\
    Maastricht University \\
    \texttt{Network Science}
}

\date{\today}

% ------------------------
% DOCUMENT
% ------------------------
\begin{document}

\twocolumn[
\maketitle
\begin{abstract}
Food webs represent the trophic interactions among species within an ecosystem. 
This paper investigates methods for identifying key species using network analysis tools such as centrality metrics, trophic levels, and structural importance. 
We apply these methods to a dataset obtained from iNaturalist to highlight the role of influential species within the community.
\end{abstract}
\vspace{0.5cm}
]

% ------------------------
\section{Introduction}
Food webs describe who eats whom in an ecosystem. Identifying key species helps ecologists understand ecosystem resilience, stability, and energy flow. 

This paper explores network-theoretic methods to detect influential species, such as:
\begin{itemize}
    \item Degree centrality
    \item Betweenness centrality
    \item Keystone species detection
    \item Community structure
\end{itemize}

% ------------------------
\section{Background and Related Work}
Provide ecological background: trophic levels, predator-prey dynamics, ecological networks.

Cite prior work on network analysis in ecology.

% ------------------------
\section{Dataset}

The data used in this paper comes from the iNaturalist project “Who Eats Whom”.
This project collects predator and prey interactions that are uploaded by active users in the iNaturalist community. 
To access the dataset, it is necessary to create an account on the platform.

In total, the project contains nearly 13,000 observations, where each observation represents an interaction between two 
species, one acting as the predator and the other as the prey.

Before downloading the dataset, iNaturalist allows users to select the features they want to export.
For this paper, we downloaded the following fields:

\begin{itemize}
\item Observation ID
\item Scientific name of the species
\item Common name of the species
\item Taxon name
\item Interaction type (whether the species is eating or being eaten)
\item Partner species ID
\end{itemize}

Because the dataset does not provide much biological information beyond taxonomic identifiers, we created a script that queries the Gemini API to gather additional ecological attributes for each species.
The information requested from the API included:

\begin{itemize}
\item Average weight of the species
\item Average size of the species
\item Diet classification (herbivore, carnivore, omnivore, insectivore)
\item Average life span
\item Typical habitats
\item Continents where the species is commonly found
\end{itemize}

The original dataset from iNaturalist was mainly used to construct the food web and perform the first exploratory analyses.
The augmented dataset, enriched with biological features from the Gemini API, allowed us to train a Graph Neural Network (GNN) that helped predict potential missing links in the food web.
The details of this model and its performance are described in later sections.

% ------------------------
\section{Methods}
Explain the network modeling approach:
\begin{itemize}
    \item Directed/undirected representation
    \item Weighted or unweighted interactions
    \item Defined metrics: degree, betweenness, PageRank, etc.
\end{itemize}

Optionally include equations:
\[
C_B(v) = \sum_{s \neq v \neq t} \frac{\sigma_{st}(v)}{\sigma_{st}}
\]

% ------------------------
\section{Results}
Include figures:
\begin{figure}[h]
    \centering
    \includegraphics[width=0.45\textwidth]{images/community_modules.png}
    \caption{Community modules.}
\end{figure}

Summarize key findings:
\begin{itemize}
    \item Which species are most central?
    \item Are there keystone species?
\end{itemize}

% ------------------------
\section{Discussion}
Interpret the results:
\begin{itemize}
    \item Ecological implications
    \item Limitations of the dataset
    \item Whether centrality = ecological importance?
\end{itemize}

% ------------------------
\section{Conclusion}
Summarize your goals and findings. Suggest future work, dataset improvements, or validation with field ecology.

% ------------------------
\section*{Acknowledgments}
(Optional) Thank advisors, data providers, open-source tools.

% ------------------------
\bibliographystyle{plain}
\bibliography{references}

\end{document}